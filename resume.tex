%%%%%%%%%%%%%%%%%%%%%%%%%%%%%%%%%%%%%%%%%
% Medium Length Professional CV
% LaTeX Template
% Version 2.0 (8/5/13)
%
% This template has been downloaded from:
% http://www.LaTeXTemplates.com
%
% Original author:
% Trey Hunner (http://www.treyhunner.com/)
%
% Important note:
% This template requires the resume.cls file to be in the same directory as the
% .tex file. The resume.cls file provides the resume style used for structuring the
% document.
%
%%%%%%%%%%%%%%%%%%%%%%%%%%%%%%%%%%%%%%%%%

%----------------------------------------------------------------------------------------
%	PACKAGES AND OTHER DOCUMENT CONFIGURATIONS
%----------------------------------------------------------------------------------------

\documentclass{resume} % Use the custom resume.cls style

\usepackage[left=0.75in,top=0.6in,right=0.75in,bottom=0.6in]{geometry} % Document margins

\name{Cijo George} % Your name
%\address{B-301, Samvriddhi Gardenia Apartments} % Your address
%\address{Sahakar Nagar, Bangalore - 560092} % Your secondary addess (optional)
\address{{\bf Email:} cijogeorge.in@gmail.com, {\bf Phone:} +91 99 4567 0410}  
\address{{\bf LinkedIn}.com/in/cijogeorge, {\bf GitHub}.com/cijogeorge}

\begin{document}

%----------------------------------------------------------------------------------------
%	PROFILE SUMMARY
%----------------------------------------------------------------------------------------

\begin{rSection}{Profile Summary}

Data Scientist/ Machine Learning Engineer with interest and experience in statistical analysis and machine learning techniques, frameworks and platforms. Passionate about designing and developing end-to-end data-driven solutions to challenging problems using state-of-the-art/ emerging technologies. 

\end{rSection}

%----------------------------------------------------------------------------------------
%       TECHNICAL STRENGTHS SECTION
%----------------------------------------------------------------------------------------

\begin{rSection}{Technical Skills}

\begin{rTechExpertiseList}
\item {\bf Statistical analysis and machine learning}: Time-series analysis, forecasting and anomaly detection, classification, clustering, feature engineering, dimensionality reduction, model validation and evaluation techniques, intuitive understanding of mathematics behind learning models.
\item {\bf Programming languages}: Python (scikit-learn, pandas, flask et al.), R (caret, dplyr, forecast, shiny et al.), BASH scripting (sed, awk et al.), SQL, C, C++, MPI, CUDA, Perl, PHP, HTML
\item {\bf Frameworks, platforms and tools}: Hadoop (Hive, MapReduce), Jupyter, Docker, Nginx, Gunicorn, Git, Amazon Web Services (AWS), Cask Data Application Platform (CDAP), LaTeX.
\item {\bf Other skills}: RESTful API design, agile development, working knowledge of operating systems, storage systems and networks.
\end{rTechExpertiseList}

\end{rSection}

%----------------------------------------------------------------------------------------
%	WORK EXPERIENCE SECTION
%----------------------------------------------------------------------------------------

\begin{rSection}{Professional Experience}

\begin{rSubsection}{NetApp}{August 2012 - Present}{Member Technical Staff/ Data Scientist, Advanced Technology Group}{Bangalore}

\item Lead advanced development projects with responsibility of studying the problem, proposing a solution, developing a working prototype/ proof-of-concept and taking prototypes into production.
\smallskip
\item Successfully completed multiple big data analytics projects involving statistical analysis and machine learning techniques including {\em time-series analysis/ anomaly detection/ forecasting}, {\em classification}, {\em clustering}, {\em feature engineering} and {\em dimensionality reduction} on machine generated data.
\smallskip
\item Designed and developed software to deploy statistical and machine learning models in production, with scalable {\em RESTful API interfaces} and {\em scale-out processing} of data analysis jobs for specific use-cases.
\item Designed and prototyped a {\em data virtualization} solution providing a {\em data-pipeline} based approach for community-driven development of a semantic data catalog, with the goal of absorbing common machine learning and advanced analytics based data transformations into the data platform layer.
\smallskip
\item Co-authored technical reports on emerging technologies in big data analytics/ machine learning.
\smallskip
\item Co-authored an external publication on identifying/ troubleshooting performance problems in storage systems based on anomaly detection on time-series metrics collected from live systems.
\smallskip
\item Involved in pitching new project ideas, interacting with stakeholders (technical and non-technical), mentoring juniors/ interns and collaborating with researchers from top Indian universities.

\end{rSubsection}

%------------------------------------------------

\begin{rSubsection}{Nokia Siemens Networks}{June 2009 - July 2010}{Software Engineer, R\&D}{Bangalore}
\item Part of development team for HLR (Home Location Register), a key component for call handling and value added services in the telecommunications {\em core network}.
\smallskip
\item Involved in feature development for the next generation HLR release versions as well as maintenance of older versions.
\end{rSubsection}

\end{rSection}

%----------------------------------------------------------------------------------------
%	EDUCATION SECTION
%----------------------------------------------------------------------------------------

\begin{rSection}{Education and Research Experience}

{\bf Indian Institute of Science, Bangalore} \hfill {\em June 2013} \\ 
{\em Supercomputer Education and Research Centre} \smallskip \\
{\em Degree:} {\bf Master of Science in Engineering (by research)} \\
{\em Thesis Title:} Fault Tolerance Strategies for Large Scale Systems

\begin {rEduContList}

\item Developed and prototyped a fault avoidance technique for MPI applications based on partial replication of processes and dynamic changing of replica set, using failure probability models for predicting impending failures.
\smallskip
\item Developed a fault avoidance technique for malleable applications that dynamically changes the fault tolerance mechanism of the application throughout its execution, using failure probability models for predicting impending failures.
\smallskip
\item Developed a generic fault tolerance simulator for parallel computing systems. 
\smallskip
\item {\em Course project:} Developed an adaptive technique to maximize resource utilization of physical systems in a Cloud environment based on resource utilization metrics.
\smallskip
\item {\em Course project:} Worked on implementation of thread scheduling, system call interface, virtual memory and file system in Pintos OS framework.
\smallskip
\item {\em Teaching assistantship:} Worked as a TA for two graduate level courses - High Performance Computing, Parallel Programming.

\end {rEduContList}

{\bf Cochin University of Science and Technology} \hfill {\em April 2009} \\
{\em Government Model Engineering College} \smallskip \\
{\em Degree:} {\bf Bachelor of Technology, Computer Science and Engineering}

\begin {rEduContList}
\item {\em Course projects:} USB mass storage device sharing over TCP/IP network; A collaborative text editor.
\item {\em Pet projects:} An online stock-exchange simulator with real-time data feeds from National Stock Exchange; BASH-as-a-service: A web-browser based remote interface to BASH.
\end {rEduContList}

\end{rSection}

%----------------------------------------------------------------------------------------
%	PUBLICATION
%----------------------------------------------------------------------------------------

\begin{rSection}{External Publication}

\begin{rPubPatContList}
\item {\bf Cijo George}, Sathish Vadhiyar, ``Fault Tolerance on Large Scale Systems using Adaptive Process Replication", {\em IEEE Transactions on Computers}, September 2014.
\item Vipul Mathur, {\bf Cijo George}, Jayanta Basak, ``Anode: Empirical Detection of Performance Problems in Storage Systems using Time-Series Analysis of Periodic Measurements", {\em MSST '14: Proceedings of the 30th International Conference on Massive Storage Systems and Technology (IEEE)}, Santa Clara, California, June 2014.
\item {\bf Cijo George}, Sathish Vadhiyar, ``AdFT: An Adaptive Framework for Fault Tolerance on Large Scale Systems using Application Malleability", {\em ICCS '12: Proceedings of the International Conference on Computational Science (Elsevier)}, Omaha, Nebraska, June 2012.
\end{rPubPatContList}
   
\end{rSection}

%----------------------------------------------------------------------------------------
%	PATENTS
%----------------------------------------------------------------------------------------

\begin{rSection}{Patents Filed}

\begin{rPubPatContList}
\item Vipul Mathur, {\bf Cijo George}, Swaminathan Ramany, {\em ``System and Method for Analyzing a Storage System for Performance Problems using Parametric Data"}, United States 14/194,467, Filed: February 28, 2014.
\end{rPubPatContList}
   
\end{rSection}

%----------------------------------------------------------------------------------------
%	AWARDS & RECOGNITION
%----------------------------------------------------------------------------------------

%\begin{rSection}{Awards and Recognition}

%\begin{rAwardsContList}
%\item {\em CTO Awards} at NetApp - 2013, 2016.
%\item {\em Spot Awards} at NetApp - 2016, 2017.
%\item {\em Thank You} award for {\em Remarkable Contribution} at Nokia Networks - 2010.
%\end{rAwardsContList}
   
%\end{rSection}

%----------------------------------------------------------------------------------------
%	SOCIAL MEDIA
%----------------------------------------------------------------------------------------

\begin{rSection}{Contact Details and Social Profiles}

\begin{rPubPatContList}
\item {\bf Email:} cijogeorge.in@gmail.com, {\bf Phone:} +91 99 4567 0410
\item {\bf LinkedIn}.com/in/cijogeorge, {\bf GitHub}.com/cijogeorge
\end{rPubPatContList}

\end{rSection}

%----------------------------------------------------------------------------------------
%	EXAMPLE SECTION
%----------------------------------------------------------------------------------------

%\begin{rSection}{Section Name}

%Section content\ldots

%\end{rSection}

%----------------------------------------------------------------------------------------

\end{document}
